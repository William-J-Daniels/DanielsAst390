\documentclass[12pt, letterpaper]{article}

\usepackage{geometry}

\usepackage{float, graphicx}
    \graphicspath{ {image/} }

\usepackage{amsmath, amssymb, amsthm, cases, enumitem, nicefrac}
    \allowdisplaybreaks
    \theoremstyle{definition}
    \newtheorem{dfn}{Definition}
    \newtheorem{lem}{Lemma}
    \newtheorem{thm}{theorem}
    \newtheorem{cor}{corollary}
    \newtheorem{prp}{Proposition}

\usepackage{tikz}

\usepackage{enumitem}

\usepackage{caption, subcaption}

\usepackage[toc]{multitoc}
    \setlength{\columnseprule}{0.5pt}

\usepackage{geometry}
    \geometry{letterpaper, margin = 1in}

\usepackage{hyperref}
    \hypersetup{
        colorlinks = true,
        allcolors = blue
    }


\begin{document}

\noindent
\begin{minipage}{0.5\textwidth}
    Will Daniels; 112774725

    May 4, 2023

    Computational Astrophysics; AST 390

    Final project
\end{minipage}
%
\begin{minipage}{0.5\textwidth}
    \begin{flushright}
        \includegraphics[height = 48pt]{../../LatexAssets/SBULogoStacked}
    \end{flushright}
\end{minipage}
\noindent
\rule{\textwidth}{1pt}

\tableofcontents

\section{Introduction}
Symplectic integrators are those which are able to conserve the hamiltonian,
\(\mathcal{H}\), of a system. They are useful in a variety of simulation
settings, especially long-term simulations, such as procession or stellar
evolution codes, where an unboundedly changing energy will result in unusable
results.

\section{Yoshida symplectic integration scheme}
In 1990, Yoshida published a method for finding symplectic integrators of
arbitrary even order. He realized that when the hamiltonian is seperable and
representative of the total conserived energy, \(\mathcal{H} =
\mathcal{T}(p) + \mathcal{V}(q)\), that through operator relations, we can
solve the following system of non-linear equations for the coefficients \(c_i\)
and \(d_i\) to construct a symplectic integrator of even order \(n\).
\begin{equation}
  \exp(\tau (A + B)) = \prod_{i=1}^k \exp(C_i \tau A) \exp(d_i \tau B)
  + \mathcal{O}(\tau^{n+1})
\end{equation}
Here, \(A\) and \(B\) are non-commutative operators and \(\tau\) is a small
real number. The \(c_i\) and \(d_i\) are also real.

In priciple, this enables us to contruct arbitrarily high order even symplectic
integrators. Practically, this may not be desireable because higher orders
require increasingly many (potentially expensive) acceleration updates.
Furthermore, the challange assosiated with solving the non linear system of
equations makes doing so prohibative. In his paper, Yoshida comments that
Newton-Rhapson is not applicable because constructing the Jacobian matrix is
also difficult. He used the Brent method to provide multiple solutions for
integrators of order 6 and 8.

\section{Compared to other sympletic integrators}
For the coding portion of my project, I decided to compare Yoshida4 to the
symplectic integrators we learned in the course, Euler Cromer (first oder) and
Verlet (second order). I did so by simulating the orbit of a 1 SM planet about
a 1 SM star, with a perihelion of 1 AU and eccentricity 0.7.
The orbit and energy of each integrator over 64 periods is shown in the figure.
Clearly, the higher order Yoshida4 is superior to Verlet and EulerCromer.

\begin{figure}[h]
  \centering
  \caption{Comparison of integration schemes}
  \begin{subfigure}{0.475\textwidth}
    \includegraphics[width=\textwidth]{../data/Orbits.jpeg}
  \end{subfigure} \hfill \begin{subfigure}{0.475\textwidth}
    \includegraphics[width=\textwidth]{../data/Energies.jpeg}
  \end{subfigure}
\end{figure}

\end{document}
