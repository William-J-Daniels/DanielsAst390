\documentclass[12pt, letterpaper]{article}

\usepackage{geometry}

\usepackage{float, graphicx}
    \graphicspath{ {image/} }

\usepackage{amsmath, amssymb, amsthm, cases, enumitem, nicefrac}
    \allowdisplaybreaks
    \theoremstyle{definition}
    \newtheorem{dfn}{Definition}
    \newtheorem{lem}{Lemma}
    \newtheorem{thm}{theorem}
    \newtheorem{cor}{corollary}
    \newtheorem{prp}{Proposition}

\usepackage{tikz}

\usepackage{enumitem}

\usepackage{caption, subcaption}

\usepackage[toc]{multitoc}
    \setlength{\columnseprule}{0.5pt}

\usepackage{geometry}
    \geometry{letterpaper, margin = 1in}

\usepackage{hyperref}
    \hypersetup{
        colorlinks = true,
        allcolors = blue
    }


\begin{document}

\noindent
\begin{minipage}{0.5\textwidth}
    Will Daniels; 112774725

    March 12, 2023

    Computational Astrophysics; AST 390

    Homework 3
\end{minipage}
%
\begin{minipage}{0.5\textwidth}
    \begin{flushright}
        \includegraphics[height = 48pt]{../../LatexAssets/SBULogoStacked}
    \end{flushright}
\end{minipage}
\noindent
\rule{\textwidth}{1pt}

\tableofcontents

\section{Euler and Euler-Cromer methods on a simple pendulum}

\begin{figure}[!b]
  \centering
  \caption{Comparison of Euler and Euler-Cromer methods for a pendulum starting
    from rest at 10 and 100 degrees using a time step of 0.1 seconds}
  \label{fig:pend}
  \includegraphics[width=\textwidth]{../data/P1Plots}
\end{figure}

Figure \ref{fig:pend} displays the behavior of two methods for the specified
pendulum. We expect the Euler method to cause the amplitude  to tend towards
infinity but that is not what we observe. I was unable to produce this behavior
for any combination of time step or run time that did not also cause the same to
occur with the Euler-Cromer. It is unclear to me if that is because my
implementation has a mistake or because I was unable to find a good set of
parameters.

The period of the Euler and Euler-Cromer methods seem to agree for the same
initial conditions; for 10 degrees, we observe a period of about 6 seconds and
for 100 degrees we observe a period of about 8 seconds.

\section{Velocity-verlet method on a simple pendulum}

Table \ref{tab:pend} shows the ratio of the initial energy to the energy after a
number of periods for the velocity-verlet pendulum. Since the velocity-verlet
method is symplectic, the result that the energy is the same when varying the
time step is expected.

\begin{table}[h]
  \centering
  \caption{Convergence of energy for the velcity-verlet method}
  \label{tab:pend}
  \begin{tabular}{c c}
    \hline\hline
    Time step [s] & \(\nicefrac{E_0}{E_t}\) \\
    \hline
    0.5     & 1.00 \\
    0.25    & 1.00 \\
    0.125   & 1.00 \\
    0.0625  & 1.00 \\
    0.03125 & 1.00 \\
    \hline\hline
  \end{tabular}
\end{table}

\end{document}
