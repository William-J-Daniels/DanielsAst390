\documentclass[12pt, letterpaper]{article}

\usepackage{geometry}

\usepackage{float, graphicx}
    \graphicspath{ {image/} }

\usepackage{amsmath, amssymb, amsthm, cases, enumitem, nicefrac}
    \allowdisplaybreaks
    \theoremstyle{definition}
    \newtheorem{dfn}{Definition}
    \newtheorem{lem}{Lemma}
    \newtheorem{thm}{theorem}
    \newtheorem{cor}{corollary}
    \newtheorem{prp}{Proposition}

\usepackage{tikz}

\usepackage{enumitem}

\usepackage{caption, subcaption}

\usepackage[toc]{multitoc}
    \setlength{\columnseprule}{0.5pt}

\usepackage{geometry}
    \geometry{letterpaper, margin = 1in}

\usepackage{hyperref}
    \hypersetup{
        colorlinks = true,
        allcolors = blue
    }


\begin{document}

\noindent
\begin{minipage}{0.5\textwidth}
    Will Daniels; 112774725

    February 26, 2023

    Computational Astrophysics; AST 390

    Homework 2
\end{minipage}
%
\begin{minipage}{0.5\textwidth}
    \begin{flushright}
        \includegraphics[height = 48pt]{../../LatexAssets/SBULogoStacked}
    \end{flushright}
\end{minipage}
\noindent
\rule{\textwidth}{1pt}

\tableofcontents

\section{Composite Simpson's rule for odd bins}
The modifications of the composite Simpon's rule for integration was found
using Sympy in the Jupyter notebook in \verb|tools/|. I found the area under
the last two points of the parabola fitted by the last three points to be
\begin{equation}
  \frac{dx}{12} \left( 5f_0 + 8f_1 - f_2 \right)
\end{equation}
Table \ref{tab:converge} describes the convergence of my implimentation. I
extended the number of bins to better see the expected behavior of fourth order
convergence.

\begin{table}[!b]
    \centering
    \caption{Convergence of composite Simpson's rule with odd numbers of bins}
    \label{tab:converge}
    \begin{tabular}{ c c c }
    \hline\hline
    Bins & Error & Ratio\\
    \hline
    3 & 5.006 & -- \\
    5 & 0.796 & 6.3 \\
    9 & 1.051 & 0.75 \\
    17 & 0.006 & 170 \\
    33 & 0.02 & 0.28 \\
    65 & 0.002 & 12 \\
    129 & 0.0001 & 15 \\
    257 & 0.00008 & 16 \\
    513 & 0.0000005 & 16 \\
    1025 & 0.00000003 & 16 \\
    \hline\hline
    \end{tabular}
\end{table}

\section{Maxwell-Boltzmann distribution integration}
Next we are asked to use our Simpson's rule implimentations to integrate the
Maxwell-Boltzmann distribution for the condotions at the center of the Sun.

Before performing the integration, it is desireable to make a substitution in
the integral.
\begin{align}
  \left< v \right> &=
  \frac{1}{n_I} \int_0^\infty dp\,
  \frac{n_I}{\left(2 \pi m_I k_B T\right)^{\nicefrac{3}{2}}}
  \exp \left(\frac{-p^2}{2 m_I k_B T}\right)
  4 \pi p^2 \frac{p}{m_I}
  \\
  &= \frac{1}{\left(2 \pi m_I k_B T\right)^{\nicefrac{3}{2}}}
  \frac{4 \pi}{m_I}
  \int_0^\infty dp\, \exp \left(\frac{-p^2}{2 m_I k_B T}\right)
  p^3
  \intertext{Here we make the substitution
    \(x = \frac{p}{\sqrt{2 m_I k_B T}}\). This gives us
    \(p = x \sqrt{2 m_I k_B T}\), \(dp = \sqrt{2 m_I k_B T} \,dx\), and leaves
    the bounds of the integral unchanged.}
  &= \frac{\sqrt{2 m_I k_B T}}{\left(2 \pi m_I k_B T\right)^{\nicefrac{3}{2}}}
  \frac{4 \pi}{m_I}
  \int_0^\infty dx\, \exp\left(-x^2\right) x^3
  \left( 2 m_I k_B T\right)^{\nicefrac{3}{2}}
  \\
  &= \frac{\sqrt{2 m_I k_B T}}{\pi^{\nicefrac{3}{2}}} \frac{4 \pi}{m_I}
  \int_0^\infty dx\, \exp\left(-x^2\right) x^3
  \\
  &= 4 \sqrt\frac{2 k_B T}{m_I \pi}
  \int_0^\infty dx\, \exp\left(-x^2\right) x^3
\end{align}
The integral evaluates to \(\frac{1}{2}\) and this expression matches the
analytic solution of \(2\sqrt\frac{2 k_B T}{m_I \pi}\). Integrating this using
the transformations described in lecture, we find that the numeric and analytic
solutions yeild 561511 meters per second, with the numerical integration taking
24 iterations to converge.

\end{document}
