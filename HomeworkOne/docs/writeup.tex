\documentclass[12pt, letterpaper]{article}

\usepackage{geometry}

\usepackage{float, graphicx}
    \graphicspath{ {image/} }

\usepackage{amsmath, amssymb, amsthm, cases, enumitem, nicefrac}
    \allowdisplaybreaks
    \theoremstyle{definition}
    \newtheorem{dfn}{Definition}
    \newtheorem{lem}{Lemma}
    \newtheorem{thm}{theorem}
    \newtheorem{cor}{corollary}
    \newtheorem{prp}{Proposition}

\usepackage{tikz}

\usepackage{enumitem}

\usepackage{caption, subcaption}

\usepackage[toc]{multitoc}
    \setlength{\columnseprule}{0.5pt}

\usepackage{geometry}
    \geometry{letterpaper, margin = 1in}

\usepackage{hyperref}
    \hypersetup{
        colorlinks = true,
        allcolors = blue
    }


\begin{document}

\noindent
\begin{minipage}{0.5\textwidth}
    Will Daniels; 112774725

    February 10, 2023

    Computational Astrophysics; AST 390

    Homework 1
\end{minipage}
%
\begin{minipage}{0.5\textwidth}
    \begin{flushright}
        \includegraphics[height = 48pt]{SBULogoStacked}
    \end{flushright}
\end{minipage}
\noindent
\rule{\textwidth}{1pt}

\tableofcontents

\section{Second derivative}
Consider the taylor expansions
\begin{align}
    f_{i+1} &=
    f_i +
    \Delta x \left. \frac{df}{dx} \right|_i +
    \frac{1}{2} \Delta x^2 \left. \frac{d^2}{dx^2} \right|_i +
    \mathcal{O} \left(\Delta x^3 \right)
    \\
    f_{i-1} &=
    f_i -
    \Delta x \left. \frac{df}{dx} \right|_i +
    \frac{1}{2} \Delta x^2 \left. \frac{d^2}{dx^2} \right|_i +
    \mathcal{O} \left(\Delta x^3 \right)
\end{align}
To find the second order approximation of the first derivative, we
subtracted the two expressions. Now we add them so the second derivative
term doesn't cancel.
\begin{align}
    f_{i+1} + f_{i-1}
    &=
    2 f_i +
    \Delta x^2 \frac{d^2f}{dx^2} +
    \mathcal{O} \left( \Delta x^3 \right)
    \\
    \frac{d^2f}{dx^2}
    &=
    \frac{f_{i+1} - f_i + f_{i-1}}{\Delta x^2} +
    \mathcal{O} \left( \Delta x^3 \right)
\end{align}

As expected, the algorithm converges quadraticly. Ten iterations are presented
in table \ref{tab:converge}.

\begin{table}[!b]
    \centering
    \caption{Ten iterations of the algorithm. The initial step size is 0.5 and
        it reduces by a factor of two each time}
    \label{tab:converge}
    \begin{tabular}{ c c c }
    \hline\hline
    Iteration & Error & Ratio\\
    \hline
    1 & 0.00206261 & -- \\
    2 & 0.000518884 & 0.252 \\
    3 & 0.000129924 & 0.250 \\
    4 & 3.24936e-5 & 0.250 \\
    5 & 8.12423e-6 & 0.250 \\
    6 & 2.0311e-6 & 0.250 \\
    7 & 5.07778e-7 & 0.250 \\
    8 & 1.26944e-7 & 0.250 \\
    9 & 3.17375e-8 & 0.250 \\
    10 & 7.94152e-9 & 0.250 \\
    \hline\hline
    \end{tabular}
\end{table}

\section{Roundoff error}
Let's start by finding the equivalent expression with no floating point
subtractions.
\begin{gather}
    \sqrt{x^2 + 1} \left( \frac{\sqrt{x^2 + 1} + 1}{\sqrt{x^2 + 1} + 1} \right)
    \\
    \frac{x^2 + 1 - 1 + 1}{\sqrt{x^2 + 1} + 1}
    \\
    \frac{x^2}{\sqrt{x^2 + 1} + 1}
\end{gather}

The results of using each form are given in table \ref{tab:roundoff}. Clearly
the modified equation is more accurate.

\begin{table}[h]
  \centering
  \caption{Comparison of equivalent forms at specified inputs}
  \label{tab:roundoff}
  \begin{tabular} { c c c }
    \hline\hline
    Evaluation point & Original & Equivalent \\
    \hline
    1.0e-6 & 2 & 1e-12 \\
    1.0e-7 & 2 & 1e-14 \\
    1.0e-8 & 2 & 1e-16 \\
    \hline\hline
  \end{tabular}
\end{table}

\end{document}
