\documentclass[12pt, letterpaper]{article}

\usepackage{geometry}

\usepackage{float, graphicx}
    \graphicspath{ {image/} }

\usepackage{amsmath, amssymb, amsthm, cases, enumitem, nicefrac}
    \allowdisplaybreaks
    \theoremstyle{definition}
    \newtheorem{dfn}{Definition}
    \newtheorem{lem}{Lemma}
    \newtheorem{thm}{theorem}
    \newtheorem{cor}{corollary}
    \newtheorem{prp}{Proposition}

\usepackage{tikz}

\usepackage{enumitem}

\usepackage{caption, subcaption}

\usepackage[toc]{multitoc}
    \setlength{\columnseprule}{0.5pt}

\usepackage{geometry}
    \geometry{letterpaper, margin = 1in}

\usepackage{hyperref}
    \hypersetup{
        colorlinks = true,
        allcolors = blue
    }


\begin{document}

\noindent
\begin{minipage}{0.5\textwidth}
    Will Daniels; 112774725

    April 22, 2023

    Computational Astrophysics; AST 390

    Homework 6
\end{minipage}
%
\begin{minipage}{0.5\textwidth}
    \begin{flushright}
        \includegraphics[height = 48pt]{../../LatexAssets/SBULogoStacked}
    \end{flushright}
\end{minipage}
\noindent
\rule{\textwidth}{1pt}

\section{Finite difference on Burgers' equation}

For this assigment, we consider two first-order finite-difference techniques for
solving Burgers' equation, \( u_t + u u_x = 0 \). The two analytically identical
forms are
\begin{equation}
  u_i^{n+1} = u_i^n -
  \frac{\Delta t}{\Delta x} u_i^n \left( u_i^n - u_{i-1}^n \right)
  \label{eq:method_a}
\end{equation}
and
\begin{equation}
  u_i^{n+1} = u_i^n + -
  \frac{\Delta t}{2 \Delta x} \left(
    \left( u_i^n \right)^2 - \left( u_{i-1}^n \right)^2
    \label{eq:method_b}
  \right)
\end{equation}
We will call \eqref{eq:method_a} method a and \eqref{eq:method_b} method b. The
numerical performance of the two methods is displayed in figure \ref{fig:plots}
and the shock speed tabulated in table \ref{tab:speed}. The simulation time is
0.1 seconds and the parameter \(C\) was choosen to be 0.5.

\begin{figure} [!b]
  \centering
  \caption{Plots at various number of points}
  \begin{subfigure}{0.3\textwidth}
    \centering
    \caption{32 points}
    \includegraphics[width=\textwidth]{../data/32.jpeg}
  \end{subfigure} \hfill \begin{subfigure}{0.3\textwidth}
    \centering
    \caption{64 points}
    \includegraphics[width=\textwidth]{../data/64.jpeg}
  \end{subfigure} \hfill \begin{subfigure}{0.3\textwidth}
    \centering
    \caption{128 points}
    \includegraphics[width=\textwidth]{../data/128.jpeg}
  \end{subfigure}

  \hfill \begin{subfigure}{0.3\textwidth}
    \centering
    \caption{256 points}
    \includegraphics[width=\textwidth]{../data/256.jpeg}
  \end{subfigure} \hfill \begin{subfigure}{0.3\textwidth}
    \centering
    \caption{512 points}
    \includegraphics[width=\textwidth]{../data/512.jpeg}
  \end{subfigure} \hfill
  \label{fig:plots}
\end{figure}

\begin{table}
  \centering
  \caption{Shock speeds}
  \begin{tabular}{c c c}
    \hline\hline
    Number of points & Method a & Method b \\
    \hline
    32  & 1.290 & 1.613 \\
    64  & 1.270 & 1.429 \\
    128 & 1.339 & 1.496 \\
    256 & 1.372 & 1.490 \\
    512 & 1.409 & 1.507 \\
    \hline\hline
  \end{tabular}
  \label{tab:speed}
\end{table}

We imediatly notice that method b predicts a faster shock speed than method a,
which raises the question which is correct, since the disritizations are
analytically equivalent. Although equivalent in the limit when the number of
discrete \(x\) values goes to infinity, method b has a stronger sense of
physicsal conservation laws, so we should trust it as being more accurate.

What \eqref{eq:method_b} is saying in english is that the velocity at some point
\(i\) in space after some amount of time \(dt\) is equal to that orignal
velocity minus the linearly interpolated area between the velocity at the point
of interest and the point \(dx\) behind that point scaled by a factor of
\(\frac{dt}{dx}\). This captures the flux of velocity through the discretized
space and keeps it a conserved quantity, unlike in \eqref{eq:method_a}, where no
such notion of integration between adjacent cells exits.

\end{document}
